\documentclass[11pt]{exam}
\usepackage[T1]{fontenc}
\usepackage{lmodern}
\usepackage{babel}
\usepackage{microtype}
\usepackage{amsmath,amssymb,amsthm,amsfonts,mathtools}
\usepackage{titling, float, parskip, blkarray, multirow}
\usepackage[margin=1in, lmargin=0.5in]{geometry}
\setlength{\parskip}{1ex}
\setlength{\parindent}{0in}
\usepackage[math]{cellspace}
\cellspacetoplimit 3pt
\cellspacebottomlimit 3pt
\setlength{\arraycolsep}{4pt}
\usepackage{multicol}
\usepackage{array}
\usepackage{enumerate}
\usepackage{matlab-prettifier}
\usepackage{amsmath}
\usepackage[shortlabels]{enumitem}
\usepackage{tikz}
\setlength{\droptitle}{-6em}
\pretitle{\flushleft\Large\bfseries} 
\posttitle{}
\preauthor{\flushleft\Large}
\postauthor{}
\predate{}
\postdate{\vspace{-0.3em}}
\theoremstyle{plain}
\newtheorem{theorem}{Theorem}
\newtheorem{claim}[theorem]{Claim}
\newtheorem*{claim*}{Claim}
\newtheorem{proposition}[theorem]{Proposition}
\newtheorem*{proposition*}{Proposition}
\newtheorem{corollary}[theorem]{Corollary}
\newtheorem*{corollary*}{Corollary}
\newtheorem{lemma}[theorem]{Lemma}
\newtheorem*{lemma*}{Lemma}
\theoremstyle{definition}\newtheorem{definition}[theorem]{Definition}
\theoremstyle{definition}\newtheorem*{definition*}{Definition}
\let\bar\overbar
\let\0\varnothing 
\let\phi\varphi
\let\tilde\widetilde
\def\Q{\mathbb{Q}}
\def\N{\mathbb{N}}
\def\Z{\mathbb{Z}}
\def\R{\mathbb{R}}
\def\hw{3}
\title{Math 151a, Homework \hw}
\author{Lucas Wheeler}
\date{}
\begin{document}
\maketitle
\begin{questions}
\question 2.2.9 \\
We know that the $\sin (x)$ function is continuous from $[0,2\pi]$. Furthermore we know the value of the $\sin (x)$ function will lie within $[-1,1], \forall x$. Therefore we know that $ \pi - 0.5 \leq f(x)  \leq \pi + 0.5, \forall x$ Therefore $f(x) \in [0,2\pi], \forall x$ By theorem 2.3, this implies that $f(x)$ has at least one fixed point in $[0,2\pi]$
\[f'(x) = 0.25 \cos (\frac{x}{2}) \]
$f'(x)$ exists on $(0,2\pi)$ and $|f'(x)| \leq 0.25, \forall x$. Therefore, by theorem 2.3, we can conclude that there is one unique fixed point of $f(x)$ on $(0,2\pi)$. \\
\begin{lstlisting}[style=Matlab-editor]
    function [x] = fixedpoint(a, b, tol, N0, p0)
    F = @(x) pi + 0.5 * sin(0.5*x);
    j = 1;
    p = p0;
    while j < N0
        p = F(p);
        if abs(p-p0)<tol
        % close enough to actual root, stop
            break;
        else
            p0=p;
            j = j + 1;
        end
    end
    fprintf('Iteration number = %d \n', j);
    fprintf('p = %.4f \n',p);
    fprintf('f(p) = %.4f \n', pi + 0.5 * sin(0.5*p));
\end{lstlisting}
Using the code above, to approximate the fix point to within $10^{-2}$, we get $x \approx 3.6270$ in 3 iterations if we start with a $p_0 = \pi$.\\
By corollary 2.5, we know that $|p_n - p| \leq \frac{k^n}{1-k} |p_1 - p_0|, \forall n \geq 1$. This implies that $10^{-2} \leq \frac{k^n}{1-k} |p_1 - p_0| = \frac{0.25^n}{0.75} |3.6416 - \pi| = \frac{2}{3} * 0.25^n$ 
Doing some algebra and taking the log of both sides results in an n value of $n \geq 3.0294$. This is similar to the actual iteration in which a fixed point was located. 

\question 2.2.13a
\question 2.2.13b 
\question 2.2.20 
\question 2.2.24 
\question 2.3.19 
\question 2.3.23
\question 2.3.25
\quesiton 2.3.28

\begin{parts}
\part 
\end{parts}
\end{questions}
\end{document}